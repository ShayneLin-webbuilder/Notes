% Options for packages loaded elsewhere
\PassOptionsToPackage{unicode}{hyperref}
\PassOptionsToPackage{hyphens}{url}
%
\documentclass[
]{article}
\author{}
\date{\vspace{-2.5em}}

\usepackage{amsmath,amssymb}
\usepackage{lmodern}
\usepackage{iftex}
\ifPDFTeX
  \usepackage[T1]{fontenc}
  \usepackage[utf8]{inputenc}
  \usepackage{textcomp} % provide euro and other symbols
\else % if luatex or xetex
  \usepackage{unicode-math}
  \defaultfontfeatures{Scale=MatchLowercase}
  \defaultfontfeatures[\rmfamily]{Ligatures=TeX,Scale=1}
\fi
% Use upquote if available, for straight quotes in verbatim environments
\IfFileExists{upquote.sty}{\usepackage{upquote}}{}
\IfFileExists{microtype.sty}{% use microtype if available
  \usepackage[]{microtype}
  \UseMicrotypeSet[protrusion]{basicmath} % disable protrusion for tt fonts
}{}
\makeatletter
\@ifundefined{KOMAClassName}{% if non-KOMA class
  \IfFileExists{parskip.sty}{%
    \usepackage{parskip}
  }{% else
    \setlength{\parindent}{0pt}
    \setlength{\parskip}{6pt plus 2pt minus 1pt}}
}{% if KOMA class
  \KOMAoptions{parskip=half}}
\makeatother
\usepackage{xcolor}
\IfFileExists{xurl.sty}{\usepackage{xurl}}{} % add URL line breaks if available
\IfFileExists{bookmark.sty}{\usepackage{bookmark}}{\usepackage{hyperref}}
\hypersetup{
  hidelinks,
  pdfcreator={LaTeX via pandoc}}
\urlstyle{same} % disable monospaced font for URLs
\usepackage[margin=1in]{geometry}
\usepackage{longtable,booktabs,array}
\usepackage{calc} % for calculating minipage widths
% Correct order of tables after \paragraph or \subparagraph
\usepackage{etoolbox}
\makeatletter
\patchcmd\longtable{\par}{\if@noskipsec\mbox{}\fi\par}{}{}
\makeatother
% Allow footnotes in longtable head/foot
\IfFileExists{footnotehyper.sty}{\usepackage{footnotehyper}}{\usepackage{footnote}}
\makesavenoteenv{longtable}
\usepackage{graphicx}
\makeatletter
\def\maxwidth{\ifdim\Gin@nat@width>\linewidth\linewidth\else\Gin@nat@width\fi}
\def\maxheight{\ifdim\Gin@nat@height>\textheight\textheight\else\Gin@nat@height\fi}
\makeatother
% Scale images if necessary, so that they will not overflow the page
% margins by default, and it is still possible to overwrite the defaults
% using explicit options in \includegraphics[width, height, ...]{}
\setkeys{Gin}{width=\maxwidth,height=\maxheight,keepaspectratio}
% Set default figure placement to htbp
\makeatletter
\def\fps@figure{htbp}
\makeatother
\setlength{\emergencystretch}{3em} % prevent overfull lines
\providecommand{\tightlist}{%
  \setlength{\itemsep}{0pt}\setlength{\parskip}{0pt}}
\setcounter{secnumdepth}{-\maxdimen} % remove section numbering
\ifLuaTeX
  \usepackage{selnolig}  % disable illegal ligatures
\fi

\begin{document}

\hypertarget{part-1---rsqldf-setup}{%
\section{Part 1 - Rsqldf Setup}\label{part-1---rsqldf-setup}}

\hypertarget{load-required-packages}{%
\subsubsection{1. Load Required Packages}\label{load-required-packages}}

\begin{verbatim}
install.packages("sqldf")  
library("sqldf")  
library(readxl)  # only if reading excel xlsx file.
\end{verbatim}

\hypertarget{importing-data-to-environment}{%
\subsubsection{2. Importing Data to
Environment}\label{importing-data-to-environment}}

\begin{verbatim}
orders <- read_excel("/Users/shayne_lin/Desktop/work experiennce/UCSB/PSTAT10/LunchCost/orders.xlsx")  
\end{verbatim}

\hypertarget{examples-from-sqldf-package}{%
\subsubsection{3. Examples from sqldf
package}\label{examples-from-sqldf-package}}

\begin{verbatim}
name_counts_emponly  <- sqldf("SELECT firstname, COUNT(firstname) as occurances 
                              FROM employees 
                              WHERE firstname != 'rudi' 
                              GROUP BY firstname")
\end{verbatim}

\begin{center}\rule{0.5\linewidth}{0.5pt}\end{center}

\hypertarget{part-2---sql-full-tutorial}{%
\section{PART 2 - SQL Full Tutorial}\label{part-2---sql-full-tutorial}}

\hypertarget{section-1-selecting-columns}{%
\subsection{SECTION 1: Selecting
Columns}\label{section-1-selecting-columns}}

\hypertarget{selcting-multiple-columns}{%
\subsubsection{4. Selcting Multiple
Columns}\label{selcting-multiple-columns}}

\begin{verbatim}
  To select multiple columns from a table, simply separate the column names with **commas**
\end{verbatim}

\hypertarget{select-all-columns-from-a-table}{%
\subsubsection{5. Select All Columns From a
Table}\label{select-all-columns-from-a-table}}

\begin{verbatim}
  Typing out every column name would be a pain, so there's a handy shortcut: "SELECT *":    
  
\end{verbatim}

\begin{verbatim}
  SELECT * 
  FROM people;
\end{verbatim}

\hypertarget{return-a-certain-number-of-results}{%
\subsubsection{6. Return a Certain Number of
Results}\label{return-a-certain-number-of-results}}

\begin{verbatim}
  you can use the LIMIT keyword to limit the number of rows returned:  
  
\end{verbatim}

\begin{verbatim}
  SELECT *
  FROM people
  LIMIT 10;
\end{verbatim}

\textbf{Instructions:}

Get the title, release year and country for every film.

\begin{verbatim}
  SELECT title,release_year,country
  FROM films;
\end{verbatim}

Get all columns from the films table.

\begin{verbatim}
  SELECT *
  FROM films;
\end{verbatim}

\hypertarget{select-distinct}{%
\subsubsection{7. SELECT DISTINCT}\label{select-distinct}}

If you want to select all the unique values from a column, you can use
the \textbf{DISTINCT} keyword. This might be useful if, for example,
you're interested in knowing which languages are represented in the
films table:

\begin{verbatim}
  SELECT DISTINCT language
  FROM films;
\end{verbatim}

\textbf{Instructions:}

Get all the unique countries represented in the films table.

\begin{verbatim}
  SELECT DISTINCT country
  FROM films;
\end{verbatim}

\hypertarget{learning-to-count}{%
\subsubsection{8. LEARNING TO COUNT}\label{learning-to-count}}

What if you want to count the number of employees in your employees
table? The \textbf{COUNT( )} function lets you do this by returning the
number of rows in one or more columns.

For example, this code gives the number of rows in the people table:

\begin{verbatim}
  SELECT COUNT(*)
  FROM people;
\end{verbatim}

\hypertarget{practice-with-count}{%
\subsubsection{9. Practice with COUNT}\label{practice-with-count}}

if you want to count the number of non-missing values in a particular
column, you can call \textbf{COUNT( )} on just that column. Count
returns with non-missing values;

For example, to count the number of birth dates present in the people
table:

\begin{verbatim}
  SELECT COUNT(birthdate)
  FROM people;
\end{verbatim}

It's also common to combine \textbf{COUNT( )} with \textbf{DISTINCT} to
count the number of distinct values in a column.

\begin{verbatim}
  SELECT COUNT(DISTINCT birthdate)
  FROM peoplesql
\end{verbatim}

\textbf{Instructions:}\\
Count the number of \emph{rows} in the people table.

\begin{verbatim}
  SELECT COUNT(*)
  FROM people;
\end{verbatim}

Count the number of (non-missing) birth dates in the people table.

\begin{verbatim}
  SELECT COUNT(birthdate)
  FROM people;
\end{verbatim}

Count the number of unique birth dates in the people table.

\begin{verbatim}
  SELECT COUNT(DISTINCT birthdate)
  FROM people;
\end{verbatim}

\hypertarget{section-2-filtering-rows}{%
\subsection{SECTION 2: Filtering Rows}\label{section-2-filtering-rows}}

\hypertarget{where-and}{%
\subsubsection{10. WHERE AND}\label{where-and}}

You can build up your WHERE queries by combining multiple conditions
with the AND keyword:

\begin{verbatim}
  SELECT title
  FROM films
  WHERE release_year > 1994
  AND release_year < 2000;
\end{verbatim}

OR, you can have different limitations for different fields. And You can
add as many AND conditions as you need!

\begin{verbatim}
  SELECT *
  FROM films
  WHERE release_year > 2000
  AND release_year < 2010
  AND language = 'Spanish'
\end{verbatim}

\hypertarget{where-and-or}{%
\subsubsection{11. WHERE and OR}\label{where-and-or}}

What if you want to select rows based on multiple conditions where some
but not all of the conditions need to be met? For this, SQL has the OR
operator.

For example, the following returns all films released in either 1994 or
2000:

\begin{verbatim}
  SELECT title
  FROM films
  WHERE release_year = 1994
  OR release_year = 2000;
\end{verbatim}

Note that you need to specify the column for every OR condition, so the
following is invalid:

\begin{verbatim}
  SELECT title
  FROM films
  WHERE release_year = 1994 OR 2000;
\end{verbatim}

When combining AND and OR, be sure to enclose the individual clauses in
parentheses, like so:

\begin{verbatim}
  SELECT title, release_year
  FROM films
  WHERE (release_year >= 1990 AND release_year < 2000)
  AND (language = 'French' OR language = 'Spanish')
  AND gross > 2000000;
\end{verbatim}

Otherwise, due to SQL's precedence rules, you may not get the results
you're expecting!

\hypertarget{where-in}{%
\subsubsection{12. WHERE IN}\label{where-in}}

WHERE is very useful for filtering results. However, if you want to
filter based on many conditions, WHERE can get unwieldy. For example:

\begin{verbatim}
  SELECT name
  FROM kids
  WHERE age = 2
  OR age = 4
  OR age = 6
  OR age = 8
  OR age = 10;
\end{verbatim}

So, the \emph{IN} operator will allow you to specify multiple values in
a WHERE clause, making it easier and quicker to specify multiple OR
conditions!

\begin{verbatim}
  SELECT name
  FROM kids
  WHERE age IN (2, 4, 6, 8, 10);
\end{verbatim}

\hypertarget{between}{%
\subsubsection{13. BETWEEN}\label{between}}

As you've learned, you can use the following query to get titles of all
films released in and between 1994 and 2000:

\begin{verbatim}
  SELECT title
  FROM films
  WHERE release_year >= 1994
  AND release_year <= 2000;
\end{verbatim}

Checking for ranges like this is very common, so in SQL the BETWEEN
keyword provides a useful shorthand for filtering values within a
specified range. This query is equivalent to the one above:

\begin{verbatim}
  SELECT title
  FROM films
  WHERE release_year
  BETWEEN 1994 AND 2000;
\end{verbatim}

It's important to remember that BETWEEN is inclusive, meaning the
beginning and end values are included in the results!

\hypertarget{between-and-or}{%
\subsubsection{14. BETWEEN \& AND \& OR}\label{between-and-or}}

Similar to the WHERE clause, the BETWEEN clause can be used with
multiple AND and OR operators, so you can build up your queries and make
them even more powerful! Things to note: BETWEEN AND is an inclusive
function. Here's an example:

\begin{verbatim}
  SELECT title, release_year
  FROM films
  WHERE release_year BETWEEN 1990 AND 2000
  AND budget > 100000000
  AND (language = 'Spanish' OR language= 'French');
\end{verbatim}

\hypertarget{introduction-to-null-and-is-null}{%
\subsubsection{15. Introduction to NULL and IS
NULL}\label{introduction-to-null-and-is-null}}

In SQL, NULL represents a missing or unknown value. You can check for
NULL values using the expression IS NULL. For example, to count the
number of missing birth dates in the people table:

\begin{verbatim}
  FROM people
  WHERE birthdate IS NULL;
\end{verbatim}

As you can see, IS NULL is useful when combined with WHERE to figure out
what data you're missing.

Sometimes, you'll want to filter out missing values so you only get
results which are not NULL. To do this, you can use the IS NOT NULL
operator. For example, this query gives the names of all people whose
birth dates are not missing in the people table.

\begin{verbatim}
  SELECT name
  FROM people
  WHERE birthdate IS NOT NULL;
\end{verbatim}

\hypertarget{like-and-not-like}{%
\subsubsection{16. LIKE and NOT LIKE}\label{like-and-not-like}}

As you've seen, the WHERE clause can be used to filter text data.
However, so far you've only been able to filter by specifying the exact
text you're interested in. In the real world, often you'll want to
search for a pattern rather than a specific text string.

In SQL, the LIKE operator can be used in a WHERE clause to search for a
pattern in a column. To accomplish this, you use something called a
wildcard as a placeholder for some other values. There are two wildcards
you can use with LIKE:

The \% wildcard will match zero, one, or many characters in text. For
example, the following query matches companies like `Data', `DataC'
`DataCamp', `DataMind', and so on:

\begin{verbatim}
  SELECT name
  FROM companies
  WHERE name LIKE 'Data%';
\end{verbatim}

The \_ wildcard will match a single character. For example, the
following query matches companies like `DataCamp', `DataComp', and so
on:

\begin{verbatim}
  SELECT name
  FROM companies
  WHERE name LIKE 'DataC_mp';
\end{verbatim}

You can also use the \textbf{NOT LIKE} operator to find records that
don't match the pattern you specify.

\begin{verbatim}
  SELECT name
  FROM people
  WHERE name NOT LIKE 'A%';
\end{verbatim}

\hypertarget{section-3-aggregate-functions}{%
\subsection{SECTION 3: Aggregate
functions}\label{section-3-aggregate-functions}}

Often, you will want to perform some calculation on the data in a
database. SQL provides a few functions, called aggregate functions, to
help you out with this.

For example,

\begin{verbatim}
  SELECT AVG(budget)
  FROM films;
\end{verbatim}

gives you the average value from the budget column of the films table.
Similarly, the MAX() function returns the highest budget:

\begin{verbatim}
  SELECT MAX(budget)
  FROM films;
\end{verbatim}

The SUM() function returns the result of adding up the numeric values in
a column:

\begin{verbatim}
  SELECT SUM(budget)
  FROM films;
\end{verbatim}

You can probably guess what the MIN() function does!

However, when you want to find AVG of a calculation, it is not possible.
You can find the average of an object first, then do the calculation:

Example: Get the average duration in hours for all films, aliased as
avg\_duration\_hours.

\begin{verbatim}
  SELECT AVG(duration/60.0) AS avg_duration_hours
  FROM films;
\end{verbatim}

This is wrong. Here's the correct version:

\begin{verbatim}
  SELECT AVG(duration)/60.0 AS avg_duration_hours
  FROM films;
\end{verbatim}

\hypertarget{combining-aggregate-functions-with-where}{%
\subsubsection{17. Combining aggregate functions with
WHERE}\label{combining-aggregate-functions-with-where}}

Aggregate functions can be combined with the WHERE clause to gain
further insights from your data.

For example, to get the total budget of movies made in the year 2010 or
later:

\begin{verbatim}
  SELECT SUM(budget)
  FROM films
  WHERE release_year >= 2010;
\end{verbatim}

\hypertarget{a-note-on-arithmatic}{%
\subsubsection{18. A NOTE on ARITHMATIC}\label{a-note-on-arithmatic}}

In addition to using aggregate functions, you can perform basic
arithmetic with symbols like +, -, *, and /.

So, for example, this gives a result of 12:

\begin{verbatim}
  SELECT (4 * 3);
\end{verbatim}

However, the following gives a result of 1:

\begin{verbatim}
  SELECT (4 / 3);
\end{verbatim}

What's going on here?

SQL assumes that if you divide an integer by an integer, you want to get
an integer back. So be careful when dividing!

If you want more precision when dividing, you can add decimal places to
your numbers. For example,

\begin{verbatim}
  SELECT (4.0 / 3.0) AS result;
\end{verbatim}

gives you the result you would expect: 1.333.

\hypertarget{its-as-simple-as-aliasing}{%
\subsubsection{19. It's AS simple AS
aliasing}\label{its-as-simple-as-aliasing}}

You may have noticed in the first exercise of this chapter that the
column name of your result was just the name of the function you used.
For example,

\begin{verbatim}
  SELECT MAX(budget)
  FROM films;
\end{verbatim}

gives you a result with one column, named max. But what if you use two
functions like this?

\begin{verbatim}
  SELECT MAX(budget), MAX(duration)
  FROM films;
\end{verbatim}

Well, then you'd have two columns named max, which isn't very useful!

To avoid situations like this, SQL allows you to do something called
aliasing. Aliasing simply means you assign a temporary name to
something. To alias, you use the AS keyword, which you've already seen
earlier in this course.

For example, in the above example we could use aliases to make the
result clearer:

\begin{verbatim}
  SELECT MAX(budget) AS max_budget,
         MAX(duration) AS max_duration
  FROM films;
\end{verbatim}

With calculations:

\begin{verbatim}
  SELECT title, (gross-budget) AS net_profit
  FROM films;
\end{verbatim}

\hypertarget{more-aliasing}{%
\subsubsection{20. MORE ALIASING}\label{more-aliasing}}

Practice:\\
Get the percentage of people who are no longer alive. Alias the result
as percentage\_dead. Remember to use 100.0 and not 100!

\begin{verbatim}
  SELECT COUNT(deathdate) * 100.0 / COUNT(*) AS percentage_dead
  FROM people;
\end{verbatim}

Because count only returns variables that are not null. If you count
something, you should not also specify that the thing is not null, which
will cause mistakes.

Get the number of years between the newest film and oldest film. Alias
the result as difference.

\begin{verbatim}
  SELECT MAX(release_year)-MIN(release_year) AS difference
  FROM films;
\end{verbatim}

Get the number of decades the films table covers. Alias the result as
number\_of\_decades. The top half of your fraction should be enclosed in
parentheses.

\begin{verbatim}
  SELECT (MAX(release_year) - MIN(release_year))/10 AS number_of_decades
  FROM films;
\end{verbatim}

\hypertarget{section-4-sorting-and-grouping}{%
\subsection{SECTION 4: Sorting and
Grouping}\label{section-4-sorting-and-grouping}}

\hypertarget{order-by}{%
\subsubsection{21. ORDER BY}\label{order-by}}

In SQL, the ORDER BY keyword is used to sort results in ascending or
descending order according to the values of one or more columns.

By default ORDER BY will sort in ascending order. If you want to sort
the results in descending order, you can use the DESC keyword. For
example,

\begin{verbatim}
  SELECT title
  FROM films
  ORDER BY release_year DESC;
\end{verbatim}

\hypertarget{sorting-single-columns-practice}{%
\subsubsection{22. Sorting Single Columns
Practice}\label{sorting-single-columns-practice}}

Get the title of films released in 2000 or 2012, in the order they were
released.

\begin{verbatim}
  SELECT title
  FROM films
  WHERE release_year in (2000, 2012)
  ORDER BY release_year;
\end{verbatim}

\hypertarget{sorting-multiple-columns-practice}{%
\subsubsection{23. Sorting Multiple Columns
Practice}\label{sorting-multiple-columns-practice}}

ORDER BY can also be used to sort on multiple columns. It will sort by
the first column specified, then sort by the next, then the next, and so
on. For example,

\begin{verbatim}
  SELECT birthdate, name
  FROM people
  ORDER BY birthdate, name;
\end{verbatim}

sorts on birth dates first (oldest to newest) and then sorts on the
names in alphabetical order. The order of columns is important!Remember,
to specify multiple columns you separate the column names with a comma.

\hypertarget{group-by}{%
\subsubsection{24. GROUP BY}\label{group-by}}

Now you know how to sort results! Often you'll need to aggregate
results. For example, you might want to count the number of male and
female employees in your company. Here, what you want is to group all
the males together and count them, and group all the females together
and count them. In SQL, GROUP BY allows you to group a result by one or
more columns, like so:

\begin{verbatim}
  SELECT sex, count(*)
  FROM employees
  GROUP BY sex;
\end{verbatim}

This might give, for example:

\begin{longtable}[]{@{}ll@{}}
\toprule
sex & count \\
\midrule
\endhead
male & 15 \\
female & 19 \\
\bottomrule
\end{longtable}

Commonly, GROUP BY is used with aggregate functions like COUNT() or
MAX(). Note that GROUP BY always goes after the FROM clause!

\hypertarget{group-by-practice}{%
\subsubsection{25. GROUP BY practice}\label{group-by-practice}}

As you've just seen, combining aggregate functions with GROUP BY can
yield some powerful results!

A word of warning: SQL will return an error if you try to SELECT a field
that is not in your GROUP BY clause without using it to calculate some
kind of value about the entire group.

Note that you can combine GROUP BY with ORDER BY to group your results,
calculate something about them, and then order your results. For
example,

\begin{verbatim}
  SELECT sex, count(*)
  FROM employees
  GROUP BY sex
  ORDER BY count DESC;
\end{verbatim}

might return something like:

\begin{longtable}[]{@{}ll@{}}
\toprule
sex & count \\
\midrule
\endhead
female & 19 \\
male & 15 \\
\bottomrule
\end{longtable}

because there are more females at our company than males. Note also that
ORDER BY always goes after GROUP BY.

\textbf{EXERCISES:}\\
1.Get the release year and count of films released in each year.

\begin{verbatim}
  SELECT release_year,COUNT(id)
  FROM films
  GROUP BY release_year;
\end{verbatim}

2.Get the release year and average duration of all films, grouped by
release year.

\begin{verbatim}
  SELECT release_year,AVG(duration)
  FROM films
  GROUP BY release_year;
\end{verbatim}

3.Get the IMDB score and count of film reviews grouped by IMDB score in
the reviews table.

\begin{verbatim}
  SELECT imdb_score,COUNT(id)
  FROM reviews
  GROUP BY imdb_score;
\end{verbatim}

\hypertarget{group-by-practice-2}{%
\subsubsection{26. GROUP BY practice (2)}\label{group-by-practice-2}}

Now practice your new skills by combining GROUP BY and ORDER BY with
some more aggregate functions!

Make sure to always put the ORDER BY clause at the end of your query.
You can't sort values that you haven't calculated yet!

Get the release year, country, and highest budget spent making a film
for each year, for each country. Sort your results by release year and
country.

\begin{verbatim}
  SELECT release_year,country,MAX(budget)
  FROM films
  GROUP BY release_year,country
  ORDER BY release_year,country;
\end{verbatim}

\hypertarget{where-and-aggregate-functionsavgmaxminsumgroup-by-cannot-be-together}{%
\subsubsection{27. WHERE and AGGREGATE functions:AVG,MAX,MIN,SUM,GROUP
BY cannot be
together}\label{where-and-aggregate-functionsavgmaxminsumgroup-by-cannot-be-together}}

In SQL, aggregate functions can't be used in WHERE clauses. For example,
the following query is \emph{invalid}:

\begin{verbatim}
  SELECT release_year
  FROM films
  GROUP BY release_year
  WHERE COUNT(title) > 10;
\end{verbatim}

This means that if you want to filter based on the result of an
aggregate function, you need another way! That's where the HAVING clause
comes in. For example,

\begin{verbatim}
  SELECT release_year
  FROM films
  GROUP BY release_year
  HAVING COUNT(title) > 10;
\end{verbatim}

shows only those years in which more than 10 films were released.

In how many different years were more than 200 movies released?

\begin{verbatim}
  SELECT release_year
  FROM films
  GROUP BY release_year
  HAVING COUNT(title) > 200;
\end{verbatim}

\hypertarget{all-together}{%
\subsubsection{28. All together}\label{all-together}}

Time to practice using ORDER BY, GROUP BY and HAVING together.

Now you're going to write a query that returns the average budget and
average gross earnings for films in each year after 1990, if the average
budget is greater than \$60 million.

\begin{verbatim}
  SELECT release_year, AVG(budget) AS avg_budget, AVG(gross) AS avg_gross
  FROM films
  GROUP BY release_year
  HAVING AVG(budget)> 60000000
  AND release_year > 1990
  ORDER BY avg_gross DESC;
\end{verbatim}

\hypertarget{all-together2}{%
\subsubsection{29. All together(2)}\label{all-together2}}

If you only want to return a certain number of results, you can use the
LIMIT keyword to \emph{limit} the number of rows returned

Get the country, average budget, and average gross take of countries
that have made more than 10 films. Order the result by country name, and
limit the number of results displayed to 5. You should alias the
averages as avg\_budget and avg\_gross respectively.

\begin{verbatim}
  SELECT country,AVG(budget) AS avg_budget,AVG(gross) AS avg_gross
  FROM films
  GROUP BY country
  HAVING COUNT(title) > 10
  ORDER BY country
  LIMIT 5;
\end{verbatim}

\hypertarget{a-taste-of-things-to-more-join-example}{%
\subsubsection{30. A taste of things to more-join
example}\label{a-taste-of-things-to-more-join-example}}

\begin{verbatim}
  SELECT title, imdb_score
  FROM films
  JOIN reviews
  ON films.id = reviews.film_id
  WHERE title = 'To Kill a Mockingbird';
\end{verbatim}

\hypertarget{part-3---joining-data-in-sql}{%
\section{PART 3 - JOINING DATA IN
SQL}\label{part-3---joining-data-in-sql}}

\hypertarget{section-1-introduction-to-joins}{%
\subsection{SECTION 1: Introduction to
joins}\label{section-1-introduction-to-joins}}

\hypertarget{introduction-to-inner-join}{%
\subsubsection{31. Introduction to INNER
JOIN}\label{introduction-to-inner-join}}

Basic frame of inner join:

\begin{verbatim}
  SELECT *
  FROM left_table
  INNER JOIN right_table
  ON left_table.id = right_table.id;
\end{verbatim}

example:

\begin{verbatim}
  SELECT p1.country,p1.continent,prime_minister,president
  FROM prime_misters AS p1
  INNER JOIN presidents AS p2
  ON p1.country = p2.country;
\end{verbatim}

Now, use inner join to combine \emph{cities} table with \emph{countries}
table, alias and joinning on cities.country\_code and countries.code.

\begin{verbatim}
  SELECT *
  FROM cities AS city
  INNER JOIN countries AS country
  ON cities.country_code = countries.code;
\end{verbatim}

\hypertarget{inner-join-2}{%
\subsubsection{32. INNER JOIN (2)}\label{inner-join-2}}

Making alias with inner join's table and inner join's table name;

\begin{verbatim}
  SELECT c1.name AS city, c2.name AS country
  FROM cities AS c1
  INNER JOIN countries AS c2
  ON c1.country_code = c2.code;
\end{verbatim}

Notice that to select a field in your query that appears in multiple
tables, you'll need to identify which table/table alias you're referring
to by using a . in your SELECT statement.

Exercise: get data from both the countries and economies tables to
examine the inflation rate for both 2010 and 2015.

\begin{verbatim}
  SELECT c.code AS country_code,c.name,e.year,e.inflation_rate
  FROM countries AS c
  INNER JOIN economies AS e
  ON c.code = e.code
  WHERE e.year IN (2010,2015);
\end{verbatim}

\hypertarget{inner-join--combining-more-than-two-tables-at-once}{%
\subsubsection{33. INNER JOIN- combining more than two tables at
once}\label{inner-join--combining-more-than-two-tables-at-once}}

The ability to combine multiple joins in a single query is a powerful
feature of SQL, e.g:

\begin{verbatim}
  SELECT *
  FROM left_table
  INNER JOIN right_table
  ON left_table.id = right_table.id
  INNER JOIN another_table
  ON left_table.id = another_table.id;
\end{verbatim}

As you can see here it becomes tedious to continually write long table
names in joins. This is when it becomes useful to alias each table using
the first letter of its name (e.g.~countries AS c)!

Now, for each country, you want to get the country name, its region, the
fertility rate, and the unemployment rate for both 2010 and 2015.

\begin{verbatim}
  SELECT c.code,c.name,c.region,e.year,p.fertility_rate,e.unemployment_rate
  FROM countries AS c
  INNER JOIN economies AS e
  ON c.code = e.code
  INNER JOIN populations AS p
  ON c.code = p.country_code
  WHERE e.year in (2010,2015);
\end{verbatim}

The above is wrong because of the last statement in inner join. When
selecting fields that are in multiple tables, remember to making the
connection on multiples.

explaination:The trouble with doing your last join on c.code = e.code
and not also including year is that e.g.~the 2010 value for
fertility\_rate is also paired with the 2015 value for
unemployment\_rate. Fix your previous query: in your last ON clause, use
AND to add an additional joining condition. In addition to joining on
code in c and e, also join on year in e and p.

\begin{verbatim}
  -- Select fields
  SELECT c.code, name, region, e.year, f rtility_rate, unemployment_rate
  -- From countries (alias as c)
  FROM countries AS c
  -- Join to populations (as p)
  INNER JOIN populations AS p
  -- Match on country code
  ON c.code = p.country_code
  -- Join to economies (as e)
  INNER JOIN economies AS e
  -- Match on country code and year
  ON c.code = e.code AND e.year = p.year;
\end{verbatim}

\hypertarget{inner-join-with-using}{%
\subsubsection{34. INNER JOIN with USING}\label{inner-join-with-using}}

Inner join with using\\
When joining tables with a common field name, e.g.~

\begin{verbatim}
  SELECT *
  FROM countries
  INNER JOIN economies
  ON countries.code = economies.code
\end{verbatim}

You can use USING as a shortcut:

\begin{verbatim}
  SELECT *
  FROM countries
  INNER JOIN economies
  USING(code)
\end{verbatim}

\hypertarget{self-join}{%
\subsubsection{35. SELF JOIN}\label{self-join}}

Self-joins are used to compare values in a field to other values of the
same field from within the same table.

\url{Example:finishing} off the self-join on prime\_misnisters

\begin{verbatim}
  SELECT p1.country AS country1,p2.country AS country2,p1.continent
  FROM prime_ministers AS p1
  INNER JOIN prime_minister AS p2
  ON p1.continent = p2.continent AND p1.country = p2.country
  LIMIT 13;
\end{verbatim}

\textbf{Instructions}\\
\textbf{1/3}

\begin{itemize}
\item
  Join populations with itself ON country\_code.
\item
  Select the country\_code from p1 and the size field from both p1 and
  p2. SQL won't allow same-named fields, so alias p1.size as size2010
  and p2.size as size2015.
\end{itemize}

\begin{verbatim}
  -- Select fields with aliases
  SELECT p1.country_code,
         p1.size AS size2010,
         p2.size AS size2015
  -- From populations (alias as p1)
  FROM populations AS p1
    -- Join to itself (alias as p2)
    INNER JOIN populations AS p2
      -- Match on country code
      ON  p1.country_code = p2.country_code;
\end{verbatim}

\textbf{Instructions}\\
\textbf{2/3}

\begin{itemize}
\tightlist
\item
  Notice from the result that for each \emph{country\_code} you have
  four entries laying out all combinations of 2010 and 2015.
\item
  Extend the \emph{ON} in your query to include only those records where
  the \emph{p1.year} (2010) matches with \emph{p2.year - 5} (2015 - 5 =
  2010). This will omit the three entries per \emph{country\_code} that
  you aren't interested in.
\item
  Even though you did not select it from the SELECT, as it is matching
  two completely same tables, you need to consider replication of data
  as you apply.
\end{itemize}

\begin{verbatim}
  -- Select fields with aliases
  SELECT p1.country_code,
         p1.size AS size2010,
         p2.size AS size2015
  -- From populations (alias as p1)
  FROM populations as p1
    -- Join to itself (alias as p2)
    INNER JOIN populations as p2
      -- Match on country code
      ON p1.country_code = p2.country_code
          -- and year (with calculation)
          AND p1.year = p2.year-5;
\end{verbatim}

\textbf{Instructions}\\
\textbf{3/3}

\begin{itemize}
\item
  As you just saw, you can also use SQL to calculate values like
  \emph{p2.year - 5} for you. With two fields like \emph{size2010} and
  \emph{size2015}, you may want to determine the percentage increase
  from one field to the next:
\item
  With two numeric fields and , the percentage growth from to can be
  calculated as (B-A)/A*100.0
\item
  Add a new field to \emph{SELECT}, aliased as \emph{growth\_perc}, that
  calculates the percentage population growth from 2010 to 2015 for each
  country, using \emph{p2.size} and \emph{p1.size}
\end{itemize}

\begin{verbatim}
  -- Select fields with aliases
  SELECT p1.country_code,
         p1.size AS size2010, 
         p2.size AS size2015,
         -- Calculate growth_perc
         ((p2.size - p1.size)/p1.size* 100.0) AS growth_perc
  -- From populations (alias as p1)
  FROM populations AS p1
    -- Join to itself (alias as p2)
    INNER JOIN populations AS p2
      -- Match on country code
      ON p1.country_code = p2.country_code
          -- and year (with calculation)
          AND p1.year = p2.year - 5;
\end{verbatim}

\hypertarget{case-when-and-then}{%
\subsubsection{36. Case when and then}\label{case-when-and-then}}

Often it's useful to look at a numerical field not as raw data, but
instead as being in different categories or groups.

You can use CASE with WHEN, THEN, ELSE, and END to define a new grouping
field.

\textbf{Instructions:}\\
Using the \emph{countries} table, create a new field \emph{AS
geosize\_group} that groups the countries into three groups:

\begin{itemize}
\tightlist
\item
  If \emph{surface\_area} is greater than 2 million,
  \emph{geosize\_group} is \emph{`large'}.
\item
  If \emph{surface\_area} is greater than 350 thousand but not larger
  than 2 million, \emph{geosize\_group} is \emph{`medium'}.
\item
  Otherwise, \emph{geosize\_group} is \emph{`small'}.
\end{itemize}

\begin{verbatim}
  SELECT name, continent, code, surface_area,
      -- First case
      CASE WHEN surface_area> 2000000 THEN 'large'
          -- Second case
          WHEN  surface_area > 350000 THEN 'medium'
          -- Else clause + end
          ELSE 'small' END
          -- Alias name
          AS geosize_group
  -- From table
  FROM countries;
\end{verbatim}

\hypertarget{into-to-save-edited-table-as-a-new-table}{%
\subsubsection{37. INTO to save edited table as a new
table}\label{into-to-save-edited-table-as-a-new-table}}

The table you created with the added \emph{geosize\_group} field has
been loaded for you here with the name \emph{countries\_plus}. Observe
the use of (and the placement of) the \emph{INTO} command to create this
\emph{countries\_plus} table:

\begin{verbatim}
  SELECT name, continent, code, surface_area,
      CASE WHEN surface_area > 2000000
              THEN 'large'
         WHEN surface_area > 350000
              THEN 'medium'
         ELSE 'small' END
         AS geosize_group
  INTO countries_plus
  FROM countries;
\end{verbatim}

You will now explore the relationship between the size of a country in
terms of surface area and in terms of population using grouping fields
created with \emph{CASE}.

By the end of this exercise, you'll be writing two queries back-to-back
in a single script.

\textbf{Instructions}\\
\textbf{1/3}\\
Using the \emph{populations} table focused only for the \emph{year}
2015, create a new field aliased as \emph{popsize\_group} to organize
population \emph{size} into

\begin{itemize}
\tightlist
\item
  \emph{`large'} (\textgreater{} 50 million),\\
\item
  \emph{`medium'} (\textgreater{} 1 million), and\\
\item
  \emph{`small'} groups.\\
  Select only the country code, population size, and this new
  \emph{popsize\_group} as fields.
\end{itemize}

\begin{verbatim}
  SELECT country_code,size,
    CASE WHEN size > 50000000
      THEN 'large'
         WHEN size > 1000000
      THEN 'medium'
         ELSE 'small' 
      END AS popsize_group
  FROM populations
  WHERE year = 2015;
\end{verbatim}

\textbf{Instructions}\\
\textbf{2/3}\\
* Use \emph{INTO} to save the result of the previous query as
\emph{pop\_plus}. You can see an example of this in the
\emph{countries\_plus} code in the assignment text. Make sure to include
a \emph{;} at the end of your \emph{WHERE} clause!

\begin{itemize}
\tightlist
\item
  Then, include another query below your first query to display all the
  records in \emph{pop\_plus} using SELECT * FROM pop\_plus; so that you
  generate results and this will display \emph{pop\_plus} in the query
  result.
\end{itemize}

\begin{verbatim}
  SELECT country_code,size,
      CASE WHEN size > 50000000
        THEN 'large'
           WHEN size > 1000000
        THEN 'medium'
           ELSE 'small' 
        END AS popsize_group
    INTO pop_plus
    FROM populations
    WHERE year = 2015;
    
  SELECT *
  FROM pop_plus
\end{verbatim}

\begin{itemize}
\tightlist
\item
  Keep the first query intact that creates \emph{pop\_plus} using
  \emph{INTO}.
\item
  Write a query to join \emph{countries\_plus AS c} on the left with
  \emph{pop\_plus AS p} on the right matching on the country code
  fields.
\item
  Sort the data based on \emph{geosize\_group}, in ascending order so
  that large appears on top.
\item
  Select the name, continent, geosize\_group, and \emph{popsize\_group}
  fields.
\end{itemize}

\begin{verbatim}
  SELECT country_code, size,
    CASE WHEN size > 50000000
              THEN 'large'
         WHEN size > 1000000
              THEN 'medium'
         ELSE 'small' END
         AS popsize_group
  INTO pop_plus       
  FROM populations
  WHERE year = 2015;
  
  -- Select fields
  SELECT name,continent,geosize_group,popsize_group
  -- From countries_plus (alias as c)
  FROM countries_plus AS c
    -- Join to pop_plus (alias as p)
  INNER JOIN pop_plus AS p
      -- Match on country code
    ON p.country_code=c.code
  -- Order the table    
  ORDER BY geosize_group;
\end{verbatim}

\hypertarget{section-2-outer-joins-and-cross-joins}{%
\subsection{SECTION 2: Outer joins and cross
joins}\label{section-2-outer-joins-and-cross-joins}}

Outer joins is reaching out to another table while keeping all of the
records of the original table. While inner joins keep only the records
IN both tables. Outer joins include (1) Left joins (2) Right joins (3)
Full joins.

The syntax of a LEFT JOIN:

\begin{verbatim}
SELECT p1.country, prime_minister, president
FROM prime_ministers AS p1
LEFT JOIN presidents AS p2
ON p1.country = p2.country;
\end{verbatim}

\hypertarget{left-join}{%
\subsubsection{38. Left Join}\label{left-join}}

Combine two datasets together with inner joins and left joins,
genenrally, left joins should end up with more variables at the end and
inner joins end up with less variable at the end.

\begin{verbatim}
  -- Select fields
  SELECT region, AVG(gdp_percapita) AS avg_gdp
  -- From countries (alias as c)
  FROM countries AS c
    -- Left join with economies (alias as e)
    LEFT JOIN economies as e
      -- Match on code fields
      ON c.code = e.code
  -- Focus on 2010
  WHERE year = 2010
  -- Group by region
  GROUP BY c.region
  -- Order by descending avg_gdp
  ORDER BY avg_gdp DESC;
\end{verbatim}

**Left Joins are functionally the same as right joins using reverse,
even in multiple tables' combination case.

\begin{verbatim}
  -- convert this code to use RIGHT JOINs instead of LEFT JOINs
  /*
  SELECT cities.name AS city, urbanarea_pop, countries.name AS country,
         indep_year, languages.name AS language, percent
  FROM cities
    LEFT JOIN countries
      ON cities.country_code = countries.code
    LEFT JOIN languages
      ON countries.code = languages.code
  ORDER BY city, language;
  */
  
  SELECT cities.name AS city, urbanarea_pop, countries.name AS country,
         indep_year, languages.name AS language, percent
  FROM languages
    RIGHT JOIN countries
      ON countries.code = languages.code
    RIGHT JOIN cities
      ON cities.country_code = countries.code
  ORDER BY city, language;
\end{verbatim}

\hypertarget{full-joins}{%
\subsubsection{39. Full joins}\label{full-joins}}

Full joins is a union of the combination of the sets, including both
table\_1 and table\_2 attributes at the same time.

\begin{verbatim}
  SELECT countries.name, code, languages.name AS language
  -- From languages
  FROM languages 
    -- Join to countries
    FULL JOIN countries
      -- Match on code
      USING (code)
  -- Where countries.name starts with V or is null
  WHERE countries.name LIKE 'v%' OR countries.name IS NULL
  -- Order by ascending countries.name
  ORDER BY countries.name
\end{verbatim}

\hypertarget{crossing-the-rubicon-cross-joins}{%
\subsubsection{40. CROSSING the rubicon-CROSS
JOINS}\label{crossing-the-rubicon-cross-joins}}

CROSS JOINs create all possibe combinations of two tables. It helps
generate all possible values of combinations of the table attributes.
For example, table 1 has 3 attributes and table 2 has 3 attributes,
together the make the cross joins of 9 attributes, assuming none of the
values are the same.

syntax:

\begin{verbatim}
SELECT 
FROM 
CROSS JOIN 
\end{verbatim}

\hypertarget{a-table-of-two-cities} to choose Hyderabad in both
  countries.
\item
  Select only the city name \emph{AS city} and language name \emph{AS
  language}
\end{itemize}

\begin{verbatim}
SELECT c.name AS city,l.name AS language
FROM cities AS c
CROSS JOIN languages AS l
WHERE c.name LIKE 'Hyder%';
\end{verbatim}

\hypertarget{outer-challenge}{%
\subsubsection{42. Outer challenge}\label{outer-challenge}}

In terms of life expectancy for 2010, determine the names of the lowest
five countries and their regions.

\textbf{Instructions}\\
* Select country name AS country, region, and life expectancy AS
life\_exp. * Make sure to use LEFT JOIN, WHERE, ORDER BY, and LIMIT.

\begin{verbatim}
SELECT c.name AS country,c.region,p.life_expectancy AS life_exp
FROM countries AS c
LEFT JOIN populations AS p
ON c.code = p.country_code  
WHERE p.year = 2010
ORDER BY life_exp
LIMIT 5;
\end{verbatim}

\begin{center}\rule{0.5\linewidth}{0.5pt}\end{center}

\hypertarget{section-3-set-theory-clauses}{%
\subsection{SECTION 3: Set theory
clauses}\label{section-3-set-theory-clauses}}

\hypertarget{introduction-to-set-theory-venn-theories}{%
\subsubsection{43. Introduction to Set Theory Venn
Theories}\label{introduction-to-set-theory-venn-theories}}

\textbf{Union}\\
Includes only one overlap,which means the table will only include both
of the attributes from the tables, but have everything appers only
once.\\
\textbf{Union All}\\
Includes two overlaps, which means the table will also reveal the
duplicates in the combination\\
\textbf{Basic Syntax Example for UNION}

\begin{verbatim}
SELECT prime_minister AS leader, country
FROM prime_ministers
UNION
SELECT monarch,country
FROM monarchs
ORDER BY country;
\end{verbatim}

\textbf{Note}\\
When you use \texttt{UNION} for the tables, make sure that the combined
fields should have same type of data. For example, that a character type
cannot be combined with a numeric type.

\textbf{Basic Syntax Example for UNION ALL}

\begin{verbatim}
SELECT prime_minister AS leader, country
FROM prime_ministers
UNION ALL
SELECT monarch, country
FROM monarchs
ORDER BY country
LIMIT 10;
\end{verbatim}

\textbf{Intersect}\\
Only includes the overlap once and only the overlap, only include the
intersection of two tables.\\
\textbf{Except}\\
Only includes one side of the table and exclude the overlap of the two
tables.

\hypertarget{union}{%
\subsubsection{44. Union}\label{union}}

\textbf{Instructions}\\
* Combine the two new tables into one table containing all of the fields
in economies2010.\\
* Sort this resulting single table by country code and then by year,
both in ascending order.

\begin{verbatim}
SELECT *
FROM economies2010
UNION
SELECT *
FROM economies2015
ORDER BY code,year;
\end{verbatim}

\hypertarget{union-to-count-attributes-occurrences-of-a-field-for-multiple-tables2}{%
\subsubsection{45. Union to Count Attributes Occurrences of a field for
Multiple
Tables(2)}\label{union-to-count-attributes-occurrences-of-a-field-for-multiple-tables2}}

\texttt{UNION} can also be used to determine all occurrences of a field
across multiple tables. Try out this exercise with no starter code.

\textbf{Instructions:}

\begin{itemize}
\tightlist
\item
  Determine all (non-duplicated) country codes in either the
  \texttt{cities} or the \texttt{currencies} table. The result should be
  a table with only one field called \texttt{country\_code}.\\
\item
  Sort by \texttt{country\_code} in alphabetical order.
\end{itemize}

\begin{verbatim}
SELECT c1.code
FROM currencies AS c1
UNION
SELECT c2.country_code
FROM cities AS c2
ORDER BY c2.country_code
\end{verbatim}

\hypertarget{union-all}{%
\subsubsection{46. Union all}\label{union-all}}

\begin{itemize}
\item
  As you saw, duplicates were removed from the previous two exercises by
  using \texttt{UNION}.
\item
  To include duplicates, you can use \texttt{UNION\ ALL}.
\end{itemize}

\textbf{Instructions}

\begin{itemize}
\tightlist
\item
  Determine all combinations (include duplicates) of country code and
  year that exist in either the `economies or the populations tables.
  Order by code then year.\\
\item
  The result of the query should only have two columns/fields. Think
  about how many records this query should result in.\\
\item
  You'll use code very similar to this in your next exercise after the
  video. Make note of this code after completing it.
\end{itemize}

\begin{verbatim}
  -- Select fields
  SELECT code,year
    -- From economies
    FROM economies
    -- Set theory clause
    UNION ALL
  -- Select fields
  SELECT country_code, year
    -- From populations
    FROM populations
  -- Order by code, year
  ORDER BY code, year;
\end{verbatim}

\hypertarget{introduction-to-intersect}{%
\subsubsection{47. Introduction to
Intersect}\label{introduction-to-intersect}}

\textbf{Syntax}

\begin{verbatim}
SELECT id
FROM left_one
INTERSECT
SELECT id
FROM right_one;
\end{verbatim}

\textbf{Distinction from joins and union all}

\begin{itemize}
\tightlist
\item
  \texttt{INTERSECT} looks for records in common, not individual key
  fields like what a join does to match.
\item
  \texttt{INTERSECT} will only return records that both tables have in
  common.
\end{itemize}

\textbf{Instruction for Exercise}

Finding same information from both of the tables, across the table.\\
Question: As you think about major world cities and their corresponding
country, you may ask which countries also have a city with the same name
as their country name?

\begin{verbatim}
  -- Select fields
  SELECT countries.name
    -- From countries
    FROM countries
    -- Set theory clause
    INTERSECT
  -- Select fields
  SELECT cities.name
    -- From cities
    FROM cities;
\end{verbatim}

\hypertarget{introduction-to-except}{%
\subsubsection{48. Introduction to
Except}\label{introduction-to-except}}

Remember, You saw earlier that there are some monarchs that also act as
the prime minister for their country. One way to determine those
monarchs in the monarchs table that do not also hold the title of prime
minister is to use the EXCEPT clause. This SQL query selects the monarch
field from monarchs and then looks for common entries with the
prime\_ministers field, while also keeping track of the country for each
leader. You can see in the resulting query that only the two European
monarchs are not also prime ministers in the leaders database.\\
Syntax:

\begin{verbatim}
SELECT monarch, country
FROM monarchs
EXCEPT 
SELECT prime_minister,country
FROM prime_ministers;
\end{verbatim}

Exercise: Get the names of cities in cities which are not noted as
capital cities in countries as a single field result.

Note that there are some countries in the world that are not included in
the countries table, which will result in some cities not being labeled
as capital cities when in fact they are.

Instructions 100 XP Order the resulting field in ascending order. Can
you spot the city/cities that are actually capital cities which this
query misses?

\begin{verbatim}
SELECT countries.capital
FROM countries
EXCEPT
SELECT cities.name
FROM cities
ORDER BY capital;
\end{verbatim}

\hypertarget{introduction-to-semi-joins-and-anti-joins}{%
\subsubsection{49. Introduction to Semi-joins and
Anti-joins}\label{introduction-to-semi-joins-and-anti-joins}}

\begin{itemize}
\tightlist
\item
  \textbf{Semi Join:} A semi-join matches records by key field in the
  right table with those in the left. It then picks out only the rows in
  the left table that match that condition.\\
\item
  \textbf{Anti-join:} The anti-join picks out those columns in the left
  table that do not match the condition on the right table.\\
\item
  \textbf{Difference:} In comparison to joins we learnt before, joins we
  learnt were to combine the tables. But now we are using other tables
  to give condition for our current left table.\\
  \textbf{Syntax}
\end{itemize}

Finish the semi-join (an intro to subqueries)

\begin{verbatim}
SELECT president, country, continent
FROM presidents
WHERE country IN
  (SELECT name
   FROM states
   WHERE indep_year < 1800);
\end{verbatim}

HERE's the Anti-join

\begin{verbatim}
SELECT president,country,continent
FROM presidents
WHERE continent LIKE '%America'
  AND country NOT IN
    (SELECT name
     FROM states
     WHERE indep_year < 1000);
\end{verbatim}

\hypertarget{semi-join-exercises-retrives-languages-spoken-in-the-middle-east}{%
\subsubsection{50. Semi-join Exercises-retrives languages spoken in the
Middle
East}\label{semi-join-exercises-retrives-languages-spoken-in-the-middle-east}}

\textbf{Instructions} \textbf{1/3}\\
You are now going to use the concept of a semi-join to identify
languages spoken in the Middle East. Begin by selecting all country
codes in the Middle East as a single field result using \texttt{SELECT},
\texttt{FROM}, and \texttt{WHERE}.

\begin{verbatim}
SELECT code
FROM countries
WHERE region = 'Middle East';
\end{verbatim}

\textbf{Instructions}\\
\textbf{2/3}\\
* Below the commented code, select only unique languages by name
appearing in the \texttt{languages} table. * Order the resulting single
field table by \texttt{name} in ascending order.

\begin{verbatim}
SELECT DISTINCT name
FROM languages
ORDER BY name;
\end{verbatim}

\textbf{Instructions}\\
\textbf{3/3}\\
* Combine the previous two queries into one query by adding a WHERE IN
statement to the SELECT DISTINCT query to determine the unique languages
spoken in the Middle East.\\
* Order the result by name in ascending order.

\begin{verbatim}
SELECT DISTINCT name
  FROM languages
WHERE code in
  (SELECT code
   FROM countries
   WHERE region = 'Middle East')
ORDER BY NAME;
\end{verbatim}

\hypertarget{relating-semi-jin-to-a-tweaked-inner-join.}{%
\subsubsection{51. Relating semi-jin to a tweaked inner
join.}\label{relating-semi-jin-to-a-tweaked-inner-join.}}

\textbf{Using subqueries in semi-joins:}

\begin{verbatim}
  SELECT DISTINCT name
  FROM languages
  WHERE code IN
    (SELECT code
     FROM countries
     WHERE region = 'Middle East')
  ORDER BY name;
\end{verbatim}

\textbf{Using inner join to solve the same problem}

\begin{verbatim}
SELECT DISTINCT languages.name AS language
FROM languages
INNER JOIN countries
ON languages.code = countries.code
WHERE region = 'Middle East'
ORDER BY language;
\end{verbatim}

\hypertarget{diagnosing-problems-using-anti-join}{%
\subsubsection{52. Diagnosing problems using
anti-join}\label{diagnosing-problems-using-anti-join}}

\begin{itemize}
\item
  Another powerful join in SQL is the anti-join. It is particularly
  useful in identifying which records are causing an incorrect number of
  records to appear in join queries.
\item
  You will also see another example of a subquery here, as you saw in
  the first exercise on semi-joins. Your goal is to identify the
  currencies used in Oceanian countries!
\end{itemize}

\textbf{Instructions}\\
\textbf{1/3}\\
Begin by determining the number of countries in countries that are
listed in Oceania using SELECT, FROM, and WHERE.

\begin{verbatim}
SELECT COUNT(code)
FROM countries
WHERE continent = 'Oceania';
\end{verbatim}

\textbf{Instructions}\\
\textbf{2/3}

\begin{itemize}
\tightlist
\item
  Complete an inner join with \texttt{countries\ AS\ c1} on the left and
  \texttt{currencies\ AS\ c2} on the right to get the different
  currencies used in the countries of Oceania.\\
\item
  Match \texttt{ON} the \texttt{code} field in the two tables.\\
\item
  Include the country \texttt{code}, country \texttt{name}, and
  \texttt{basic\_unit\ AS\ currency}.
\end{itemize}

Observe the query result and make note of how many different countries
are listed here.

\begin{verbatim}
SELECT c1.code,c1.name,c2.basic_unit AS currency
FROM countries AS c1
INNER JOIN currencies AS c2
USING(code)
WHERE continent = 'Oceania';
\end{verbatim}

\textbf{Instructions}\\
\textbf{3/3}\\
Note that not all countries in Oceania were listed in the resulting
inner join with currencies. Use an anti-join to determine which
countries were not included!

\begin{itemize}
\tightlist
\item
  Use \texttt{NOT\ IN} and \texttt{(SELECT\ code\ FROM\ currencies)} as
  a subquery to get the country code and country name for the Oceanian
  countries that are not included in the \texttt{currencies} table.
\end{itemize}

\begin{verbatim}
SELECT code,name
FROM countries 
WHERE continent = 'Oceania'
  AND code NOT IN
    (SELECT code 
     FROM currencies);
\end{verbatim}

\hypertarget{set-theory-challenge}{%
\subsubsection{53. Set theory challenge}\label{set-theory-challenge}}

Your task here will be to incorporate two of
\texttt{UNION}/\texttt{UNION\ ALL}/\texttt{INTERSECT}/\texttt{EXCEPT} to
solve a challenge involving three tables. In addition, you will use a
subquery as you have in the last two exercises!

\textbf{Instructions}\\
* Identify the country codes that are included in either
\texttt{economies} or \texttt{currencies} but not in
\texttt{populations}.\\
* Use that result to determine the names of cities in the countries that
match the specification in the previous instruction.

\begin{verbatim}
SELECT c1.name
FROM countries AS c1
  WHERE code IN
  (SELECT e.code
   FROM economies AS e
   AND
   SELECT c2.code
   FROM currencies AS c2
   NOT
   SELECT p.country_code
   FROM populations)
\end{verbatim}

The above is wrong. To select, use union and except in the subquery.

\begin{verbatim}
-- Select the city name
SELECT name
  -- Alias the table where city name resides
  FROM cities AS c1
  -- Choose only records matching the result of multiple set theory clauses
  WHERE country_code IN
(
    -- Select appropriate field from economies AS e
    SELECT e.code
    FROM economies AS e
    -- Get all additional (unique) values of the field from currencies AS c2   
    UNION
    SELECT c2.code
    FROM currencies AS c2
    -- Exclude those appearing in populations AS p  
    EXCEPT
    SELECT p.country_code
    FROM populations AS p
);
\end{verbatim}

\hypertarget{section-4-subqueries}{%
\subsection{SECTION 4: Subqueries}\label{section-4-subqueries}}

\hypertarget{subqueries-inside-where-and-select-clauses}{%
\subsubsection{54. Subqueries inside WHERE and SELECT
clauses}\label{subqueries-inside-where-and-select-clauses}}

\textbf{Asian countries below average \texttt{fert\_rate}}

\begin{verbatim}
SELECT name, fert_rate
FROM states
WHERE continent = 'Asia'
  AND fert_rate < 
    (SELECT AVG(fert_rate)
    FROM states);
\end{verbatim}

\textbf{Subqueries inside SELECT clausess - setup}

\begin{verbatim}
SELECT DISTINCT continent,
  (SELECT COUNT(*)
   FROM states
   WHERE prime_ministers.continent = states.continent) AS countries
FROM prime_ministers;
\end{verbatim}

\hypertarget{subquery-inside-where}{%
\subsubsection{55. Subquery inside where}\label{subquery-inside-where}}

You'll now try to figure out which countries had high average life
expectancies (at the country level) in 2015.

\textbf{Instructions} \textbf{1/2}\\
Begin by calculating the average life expectancy across all countries
for 2015.

\begin{verbatim}
SELECT AVG(life_expectancy)
FROM populations
WHERE year = 2015;
\end{verbatim}

\textbf{Instructions} \textbf{2/2}\\
Recall that you can use SQL to do calculations for you. Suppose we
wanted only records that were above \texttt{1.15\ *\ 100} in terms of
life expectancy for 2015:

\begin{verbatim}
SELECT *
  FROM populations
WHERE life_expectancy > 1.15 * 100
  AND year = 2015;
\end{verbatim}

Select all fields from \texttt{populations} with records corresponding
to larger than 1.15 times the average you calculated in the first task
for 2015. In other words, change the \texttt{100} in the example above
with a subquery.

\begin{verbatim}
-- Select fields
SELECT *
  -- From populations
FROM populations
-- Where life_expectancy is greater than
WHERE Life_expectancy >1.15 * 
(SELECT AVG(life_expectancy)
FROM populations
WHERE year = 2015) 
AND year = 2015;
\end{verbatim}

\hypertarget{subquery-inside-where-2}{%
\subsubsection{56. Subquery inside where
(2)}\label{subquery-inside-where-2}}

Use your knowledge of subqueries in \texttt{WHERE} to get the urban area
population for only capital cities.

\textbf{Instructions}

\begin{itemize}
\tightlist
\item
  Make use of the \texttt{capital} field in the \texttt{countries} table
  in your subquery.\\
\item
  Select the city name, country code, and urban area population fields.
\end{itemize}

\begin{verbatim}
SELECT cities.name,countries.code,urbanarea_pop
FROM cities
WHERE cities.name IN (
SELECT countries.capital
FROM countries
)
ORDER BY urbanarea_pop DESC;
\end{verbatim}

\hypertarget{subquery-inside-select}{%
\subsubsection{57. Subquery inside
select}\label{subquery-inside-select}}

This exercise is using either join or subquery to write queries.

You have seen previously how to use \texttt{GROUP\ BY} with aggregate
functions and an inner join to get summarized information from multiple
tables.

The code given in the first query selects the top nine countries in
terms of number of cities appearing in the cities` table. Recall that
this corresponds to the most populous cities in the world. Your task
will be to convert the second query to get the same result as the
provided code.

\textbf{Initial code:}

\begin{verbatim}
SELECT countries.name AS country, COUNT(*) AS cities_num
  FROM cities
    INNER JOIN countries
    ON countries.code = cities.country_code
GROUP BY country
ORDER BY cities_num DESC, country
LIMIT 9;
\end{verbatim}

\textbf{Transformation}

\begin{verbatim}
SELECT countries.name AS country, (
SELECT COUNT(cities.name)
FROM cities
WHERE countries.code = cities.country_code) AS cities_num
FROM countries
ORDER BY cities_num DESC, country
LIMIT 9;
\end{verbatim}

\hypertarget{subquery-inside-from-clause}{%
\subsubsection{58. Subquery inside FROM
clause}\label{subquery-inside-from-clause}}

\textbf{Example}

\begin{verbatim}
SELECT DISTINCT monarchs.continent,subquery.max_perc
FROM monarchs,
  (SELECT continent,MAX(women_parli_perc) AS max_perc
  FROM states
  GROUP BY continent) AS subquery
WHERE monarchs.continent = subquery.continent
ORDER BY continent;
\end{verbatim}

\hypertarget{subquery-inside-from}{%
\subsubsection{59. Subquery inside from}\label{subquery-inside-from}}

The last type of subquery you will work with is one inside of FROM.

You will use this to determine the number of languages spoken for each
country, identified by the country's local name! (Note this may be
different than the name field and is stored in the local\_name field.)

\textbf{Instructions}\\
\textbf{1/2}

\begin{itemize}
\tightlist
\item
  Begin by determining for each country code how many \texttt{languages}
  are listed in the languages table using \texttt{SELECT},
  \texttt{FROM}, and \texttt{GROUP\ BY}.\\
\item
  Alias the aggregated field as \texttt{lang\_num}.
\end{itemize}

My version here, which is wrong because I forgot to include code to
match, where condition\ldots{}

\begin{verbatim}
SELECT code,name, subquery.num AS lang_num
FROM languages, 
  (SELECT COUNT(name) AS num
   FROM languages
   GROUP BY code) AS subquery;
\end{verbatim}

Correct version:

\begin{verbatim}
-- Select fields
SELECT countries.local_name,subquery.lang_num
  -- From countries
  FROM countries,
    -- Subquery (alias as subquery)
    (SELECT code, COUNT(*) AS lang_num
     FROM languages
     GROUP BY code) AS subquery
  -- Where codes match
  WHERE countries.code = subquery.code
-- Order by descending number of languages
  ORDER BY lang_num DESC;
\end{verbatim}

\hypertarget{advanced-subquery}{%
\subsubsection{60. Advanced Subquery}\label{advanced-subquery}}

You can also nest multiple subqueries to answer even more specific
questions.

In this exercise, for each of the six continents listed in 2015, you'll
identify which country had the maximum inflation rate, and how high it
was, using multiple subqueries. The table result of your final query
should look something like the following, where anything between
\textless{} \textgreater{} will be filled in with appropriate values:

\begin{verbatim}
name      |    continent
\end{verbatim}

------------- \textbar{} ----------------\\
\textbar{} North America\\
\textbar{} Africa\\
\textbar{} Oceania\\
\textbar{} Europe

Again, there are multiple ways to get to this solution using only joins,
but the focus here is on showing you an introduction into advanced
subqueries.

\textbf{Instructions}\\
\textbf{1/3}

\begin{itemize}
\tightlist
\item
  Create an INNER JOIN with countries on the left and economies on the
  right with USING, without aliasing your tables or columns.\\
\item
  Retrieve the country's name, continent, and inflation rate for 2015.
\end{itemize}

\begin{verbatim}
SELECT countries.name,countries.continent,economies.inflation_rate
FROM countries
INNER JOIN economies
USING(CODE)
WHERE year = 2015;
\end{verbatim}

\textbf{Instructions}\\
\textbf{2/3}

Select the maximum inflation rate in 2015 AS max\_inf grouped by
continent using the previous step's query as a subquery in the FROM
clause.

\begin{itemize}
\tightlist
\item
  Thus, in your subquery you should:

  \begin{itemize}
  \tightlist
  \item
    Create an inner join with \texttt{countries} on the left and
    \texttt{economies} on the right with \texttt{USING} (without
    aliasing your tables or columns).\\
  \item
    Retrieve the country name, continent, and inflation rate for 2015.\\
  \item
    Alias the subquery as \texttt{subquery}.
  \end{itemize}
\end{itemize}

This will result in the six maximum inflation rates in 2015 for the six
continents as one field table. Make sure to not include
\texttt{continent} in the outer \texttt{SELECT} statement.

\hypertarget{summary-differences-between-union-and-joins}{%
\subsubsection{Summary: Differences Between Union and
Joins}\label{summary-differences-between-union-and-joins}}

\textbf{Join}

\begin{itemize}
\tightlist
\item
  Join combines data from many tables based on a matched condition
  between them.
\item
  It combines data into new columns.
\item
  Number of columns selected from each table may not be the same.
\item
  Datatypes of corresponding columns selected from each table can be
  different.
\item
  It may not return distinct columns.
\end{itemize}

\textbf{Union}

\begin{itemize}
\tightlist
\item
  SQL combines the result-set of two or more SELECT statements.
\item
  It combines data into new rows.
\item
  Number of columns selected from each table should be same.
\item
  Datatypes of corresponding columns selected from each table should be
  same.
\item
  It returns distinct rows.
\end{itemize}

\end{document}
